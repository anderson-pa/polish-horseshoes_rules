%% LyX 2.2.4 created this file.  For more info, see http://www.lyx.org/.
%% Do not edit unless you really know what you are doing.
\documentclass[11pt,letterpaper,twocolumn,english,DIV=calc]{scrartcl}
\usepackage{fontspec}
\setmainfont[Ligatures=TeX]{TeX Gyre Heros}
\setsansfont[Ligatures=TeX]{Linux Biolinum O}
\usepackage{fancyhdr}
\pagestyle{fancy}
\setcounter{secnumdepth}{2}
\setcounter{tocdepth}{2}
\usepackage{color}
\usepackage{array}
\usepackage{float}
\usepackage{units}
\usepackage{url}
\usepackage{enumitem}
\usepackage{graphicx}
\usepackage{setspace}
\usepackage[authoryear]{natbib}
\usepackage[unicode=true,
 bookmarks=true,bookmarksnumbered=false,bookmarksopen=false,
 breaklinks=false,pdfborder={0 0 0},pdfborderstyle={},backref=false,colorlinks=true]
 {hyperref}
\hypersetup{pdftitle={Polish Horseshoes, Brookline Rules},
 pdfauthor={Phillip Anderson},
 pdfsubject={polish horseshoes},
 pdfkeywords={beer, frisbee, polish, horseshoes, brookline}}

\makeatletter

%%%%%%%%%%%%%%%%%%%%%%%%%%%%%% LyX specific LaTeX commands.
% Backwards compatibility for LuaTeX < 0.90
\@ifundefined{pageheight}{\let\pageheight\pdfpageheight}{}
\@ifundefined{pagewidth}{\let\pagewidth\pdfpagewidth}{}
\pageheight\paperheight
\pagewidth\paperwidth

%% Because html converters don't know tabularnewline
\providecommand{\tabularnewline}{\\}

%%%%%%%%%%%%%%%%%%%%%%%%%%%%%% Textclass specific LaTeX commands.
\newlength{\lyxlabelwidth}      % auxiliary length 

%%%%%%%%%%%%%%%%%%%%%%%%%%%%%% User specified LaTeX commands.
%%%%%%%%%%%%%%%%%%%%%%%%%%%%%%%%%%%%%%%%%
%
%
%%%%%%%%%%%%%%%%%%%%%%%%%%%%%%%%%%%%%%%%%
% This template is based on:
%
% Large Colored Title Article LaTeX Template, Version 1.1 (25/11/12)
% Downloaded from: http://www.LaTeXTemplates.com
% Original author: Frits Wenneker (http://www.howtotex.com)
% License: CC BY-NC-SA 3.0 (http://creativecommons.org/licenses/by-nc-sa/3.0/)
%%%%%%%%%%%%%%%%%%%%%%%%%%%%%%%%%%%%%%%%%

\newcommand{\rev}{Revision 6}

%\usepackage[protrusion=true,expansion=true]{microtype} % Better typography
\usepackage[svgnames]{xcolor}
\usepackage[format=plain,labelfont=bf,up,textfont=up]{caption} % Custom captions under/above floats in tables or figures
\usepackage{booktabs} % Horizontal rules in tables
\usepackage{fix-cm}	 % Custom font sizes - used for the initial letter in the document
\usepackage{sectsty} % Enables custom section titles
\usepackage{lastpage}

\usepackage{fontspec}
\usepackage{fontawesome}

\usepackage{lettrine} % Package to accentuate the first letter of the text
\newcommand{\initial}[1]{ % Defines the command and style for the first letter
\lettrine[lines=3,lhang=0.35,nindent=0.4em]{
\color{DarkRed}
{\textsf{#1}}}{}}

\hypersetup{urlcolor=DarkRed, linkcolor=DarkRed}

\usepackage{tikz}
\usetikzlibrary{arrows,patterns}
%\setlength{\intextsep}{0cm plus1cm minus1cm}
\newcommand{\currentyear}{2019}

\fancyhf{} % clear all header and footer fields
\lfoot{\footnotesize\rev}
\cfoot{\footnotesize\copyright\ 2012-\currentyear\ by Phillip Anderson. CC BY-NC-ND}
%\rfoot{\bfseries \thepage} % except the center
\rfoot{\footnotesize Page \thepage\ of \pageref{LastPage}} % "Page 1 of 2"
\renewcommand{\headrulewidth}{0pt}
\renewcommand{\footrulewidth}{0.4pt}

\usepackage{scrextend}
\deffootnote[0.7em]{0pt}{1.6em}{\makebox[0.7em][l]{\textsuperscript\thefootnotemark}}

\usepackage{titling} % Allows custom title configuration

\newcommand{\HorRule}{\color{Black} \rule{\linewidth}{1pt}} % Defines the gold horizontal rule around the title

\pretitle{\vspace{-85pt} \begin{flushleft} \HorRule \fontsize{50}{50}  \color{DarkRed} \selectfont} % Horizontal rule before the title

\title{\textsf{Polish Horseshoes}} % Your article title

\posttitle{\par\end{flushleft}\vskip .0em} % Whitespace under the title

\preauthor{\color{DarkRed} \huge\hspace{2pt}\textsf{Brookline Rules} %\begin{flushleft}\large \lineskip 0.5em \usefont{OT1}{phv}{b}{sl} \color{DarkRed}
} % Author font configuration

%\author{Phillip Anderson} % Your name

\postauthor{\footnotesize \color{Black} % Configuration for the institution name
%University of  % Your institution

%\par\end{flushleft}
\HorRule

\hfill \color{Black}\large \rev: \today \vspace{-25pt}} % Horizontal rule after the title

\setlength{\columnsep}{5ex}
\date{} % Add a date here if you would like one to appear underneath the title block
\setlist[description]{font=\Large, leftmargin=1em}
\setlist[enumerate, 1]{label=\thesection.\arabic*}
\setlist[enumerate, 2]{leftmargin=1.4em}

\makeatother

\usepackage{polyglossia}
\setdefaultlanguage[variant=american]{english}
\begin{document}
\maketitle

\thispagestyle{fancy}

\initial{P}olish Horseshoes is an outdoor disc game where opponents
make alternating attempts to hit and defend an empty beer bottle perched
on a pole to score points, all while holding a beverage in one hand.
The game seems to have come about sometime during the early 2000's,
and is now known by a variety of alternate names such as Beersbee,
Frisbeener, and Spanish Horseshoes, among others.\footnote{\url{http://en.wikipedia.org/wiki/Polish_horseshoes}}
Many variations exist for the rules, and although they are broadly
similar on the main points, most descriptions found online are quite
broad or over-simplified. A game this good needs and deserves a solid,
common set of rules.

We first came across Polish Horseshoes late in 2010 via a friend and
were immediately hooked. \emph{Brookline Rules} is our variation of
Polish Horseshoes which we developed and fine-tuned over the next
two years in Brookline, Massachusetts and have greatly enjoyed elsewhere
in the time since. It is a staple of picnics, cookouts, and other
outdoor events whenever the weather is agreeable, and occasionally
even when it is not. Described here are our official rules, recommendations
for building your own set, and suggestions for playing safely. We
hope you enjoy the game as much as we have!

\begin{figure}[H]
\centering{}\includegraphics[width=0.22\textwidth]{images/superdisc}{\footnotesize{}Source:
\href{http://www.aerobie.com}{aerobie.com}}
\end{figure}


\part*{Equipment}

Only a few bits are needed to play, most of which are common and easy
to find. Here is the summary of what you will need.
\begin{description}
\item [{Disc:}] Almost any disc will work in a pinch, but the Aerobie®
Superdisc™ is the disc of choice.\footnote{\url{http://www.aerobie.com/products/superdisc.htm}}
Light discs tend to be cheap and difficult to control, and very heavy
discs are prone to destroying bottles very quickly. Even the larger
Aerobie® Superdisc™ Ultra, which weighs 162 g but retains the softer
edge, will have you picking up glass shards very often.
\item [{Empty~bottles~(2):}] Use standard $355\mbox{ mL}$ ($12$ U.S.
fluid ounces) bottles, and keep spares in case one breaks during play.
Empty bottles filled with expanding polyurethane foam are far less
likely to shatter and have a negligible increase in weight. Foam-filled
bottles are highly advised for play in public areas.
\item [{Plastic~cups~(2):}] $16\mbox{ oz.}$ plastic cups work well and
are easy to find. These wear out frequently so keep extras on hand
or patch with packaging tape.
\item [{Poles~(2):}] These are the most difficult pieces to source, and
may require some creativity. Poles should be $1.2\mbox{ m}$ (48'')
with one capped end to balance a bottle upon and one pointy end to
stab into the ground. Ski poles work well if a set is available, but
poles can otherwise be fashioned from $19\mbox{ mm}$ ($\nicefrac{5}{8}$'')
diameter wooden dowels, PVC pipe, or fiberglass rods. 

Tennis balls with a pole-sized hole drilled through the side make
excellent caps. The pointy end is more difficult. Dowels can simply
be sharpened, but PVC and fiberglass tubes will require manufactured
spikes that match their inner diameter. If using a tube, cut a small
slot on the end and secure the spike with a hose clamp.
\begin{figure}[!h]
\centering{}\includegraphics[width=0.8\columnwidth]{images/poleEnds-w}
\end{figure}

\item [{Tasty~beverages:}] All players will need a tasty beverage of their
choice, or something equivalent to occupy one hand during play. Glass
cups and bottles are at risk of shattering if hit so aluminum cans
are the best container. Plastic cups work as well but are more prone
to spillage.\footnote{Another option is to make a \emph{can cup} by using a can opener on
an empty beer or soda can. Can openers with finer teeth work best.
Japanese cans tend to be superior as they have thicker walls. Look
for Oi Ocha tea or get yourself a 22 oz. Sapporo.} Either way, you will be thankful every time the cup in your hand
does not explode into many tiny high-speed shards of glass.
\end{description}
\begin{figure*}[!t]
\begin{centering}
%\usetikzlibrary{arrows,patterns}
\begin{tikzpicture}[line cap=round,line join=round,>=stealth,x=1.cm,y=1.cm]

\def\pp{10} %pole to pole distance
\def\d{2} %depth of player area
\def\w{3} %width of court
\def\extra{0.3} %how much to extend guide lines
\def\mw{1} %width of pole zone
\def\md{.5} %depth of pole zone

\draw[fill=green!25] (-\pp/2-\d,-\w/2) rectangle (\pp/2+\d,\w/2);
\draw[fill=green!25,color=green!25, pattern=crosshatch dots] (-\pp/2,-\w/2) rectangle (\pp/2,\w/2);
\draw (-\pp/2-\d,-\w/2) rectangle (\pp/2+\d,\w/2);
\draw[dashed] (0,-\w/2-\extra) -- (0,\w/2+\extra) node[above] {half court};

\draw (\pp/2,0) coordinate(A);

\foreach \i in {-1,1}
  {
  \draw[rounded corners=5*\md, fill=yellow!30] (\i*\pp/2,0) ++(-\md/2,-\mw/2) rectangle ++(\md,\mw);
  \draw[dashed] (\i*\pp/2,0) +(0,-\w/2-\extra) -- +(0,\w/2+\extra) node[above] {$bp$};
  \draw[color=black,fill=brown] (\i*\pp/2 +\i*0.08,rand/3) circle (.1);
  }
\draw[thick,|<->|] (-\pp/2,-\w/2-2*\extra) -- node[fill=white] {\pp\ m} +(\pp,0);
\draw[thick,|<->|] (\pp/2,-\w/2-2*\extra) -- node[fill=white] {\d\ m} +(\d,0);
\draw[thick,|<->|] (\pp/2+\d+2*\extra,-\w/2) -- node[fill=white] {\w\ m} +(0,\w);
%\draw (0,0) circle(1);

%players
\begin{scope}[red, very thick, x=1.4mm, y=1.4mm, xshift=-5.8cm, xscale=1]
\draw (0.2,7.5) ellipse(0.5 and 0.5) ++(-110:.5) -- ++(0,-2.75);
\draw (0.,4.25) -- ++(-10:0.5) -- ++(-50:1.6) -- ++(-80:2) -- ++(-5:0.6);
\draw (0.,4.25) -- ++(-170:0.5) -- ++(-130:1.6) -- ++(-100:2) -- ++(-175:0.6);
\draw (0, 6.5) -- ++(-10:0.5) -- ++(-45:1.5) -- ++(20:1.5) -- +(70:0.65) +(0,0) -- ++(-20:.3);
\draw (0, 6.5) -- ++(-170:0.5) -- ++(-115:1.5) -- ++(0:1.5) -- +(-25:0.65) +(0,0) -- ++(30:.3);
\end{scope}
\begin{scope}[red, very thick, x=1.4mm, y=1.4mm, xshift=-6.3cm, yshift=-1.2cm, xscale=1]
\draw (0.2,7.5) ellipse(0.5 and 0.5) ++(-114:.5) -- ++(0,-2.8);
\draw (0.,4.25) -- ++(-10:0.5) -- ++(-50:1.6) -- ++(-80:2) -- ++(-5:0.6);
\draw (0.,4.25) -- ++(-170:0.5) -- ++(-130:1.6) -- ++(-100:2) -- ++(-175:0.6);
\draw (0, 6.5) -- ++(-10:0.5) -- ++(-45:1.5) -- ++(20:1.5) -- +(70:0.65) +(0,0) -- ++(-20:.3);
\draw (0, 6.5) -- ++(-170:0.5) -- ++(-135:1.5) -- ++(90:1.5) -- +(65:0.65) +(0,0) -- ++(0:.3);
\end{scope}
\begin{scope}[blue, very thick, x=1.4mm, y=1.4mm, xshift=6.5cm, xscale=-1]
\draw (0.2,7.5) ellipse(0.5 and 0.5) ++(-110:.5) -- ++(0,-2.75);
\draw (0.,4.25) -- ++(-10:0.5) -- ++(-50:1.6) -- ++(-80:2) -- ++(-5:0.6);
\draw (0.,4.25) -- ++(-170:0.5) -- ++(-130:1.6) -- ++(-100:2) -- ++(-175:0.6);
\draw (0, 6.5) -- ++(-10:0.5) -- ++(-45:1.5) -- ++(20:1.5) -- +(70:0.65) +(0,0) -- ++(-20:.3);
\draw (0, 6.5) -- ++(-170:0.5) -- ++(-115:1.5) -- ++(-10:1.5) -- +(-45:0.65) +(0,0) -- ++(20:.3);
\end{scope}
\begin{scope}[blue, very thick, x=1.4mm, y=1.4mm, xshift=5.8cm, yshift=-1.2cm, xscale=-1]
\draw (0.2,7.5) ellipse(0.5 and 0.5) ++(-114:.5) -- ++(0,-2.8);
\draw (0.,4.25) -- ++(-10:0.5) -- ++(-50:1.6) -- ++(-80:2) -- ++(-5:0.6);
\draw (0.,4.25) -- ++(-170:0.5) -- ++(-130:1.6) -- ++(-100:2) -- ++(-175:0.6);
\draw (0, 6.5) -- ++(-10:0.5) -- ++(-45:1.5) -- ++(20:1.5) -- +(70:0.65) +(0,0) -- ++(-20:.3);
\draw (0, 6.5) -- ++(-170:0.5) -- ++(115:1.5) -- ++(40:1.5) -- +(65:0.65) +(0,0) -- ++(-20:.3);
\end{scope}

\end{tikzpicture}

\vspace{-5mm}
\par\end{centering}
\caption{Top-down schematic of the court. Poles (brown dots) can be placed
anywhere within the pole zones (yellow regions). $bp$ indicates the
\emph{bottle planes}. Players stand behind the $bp$ on their respective
side.\label{fig:court}}
\end{figure*}


\part*{Court}

The court is a long, narrow clearing with a pole zone on either end,
as shown in figure \ref{fig:court}. One pole is set within each pole
zone, and players stand behind their pole, facing the opposing team.
The pole is set by planting it into the ground, followed by a cap
(if needed), inverted plastic cup, and empty bottle balanced on top
as in figure \ref{fig:pole-setup}. The cup is considered to be part
of the pole. The \emph{bottle plane} is an imaginary plane extending
vertically from the ground through the bottle and perpendicular to
the long axis of the court.\footnote{If you want to be really anal, the $bp$ should extend through the
part of the bottle closest to half court.}

Edges of the court may or may not be marked by physical boundaries
(fences, walls, windows, etc). Regions on or beyond the court boundary
deemed highly undesirable for the disc, such as windows or difficult
to access areas, are \emph{traps}. \emph{Breakers} such as rocks or
other hard objects may be located along the edges of player areas
but should not be in the court. 
\begin{figure}[!h]
\begin{centering}
\include{poleSetupCombined}\vspace{-5mm}
\par\end{centering}
\caption{Pole setup. Pointy end in the ground. The inverted plastic cup covers
the cap and supports an empty bottle. A round cap makes it easier
to balance the bottle when the pole is not vertical.\label{fig:pole-setup}}
\end{figure}


\part*{Gameplay}

\section{Starting up}
\begin{enumerate}
\item All players must have a tasty beverage or equivalent in one hand during
play.
\item Games are \emph{singles} ($1v1$) or \emph{doubles} ($2v2$).
\item The winner of a fair toss chooses to have first throw or preferred
side.
\item Both teams start with 0 points and 10 HP.
\end{enumerate}

\section{Basic play and terminology}
\begin{enumerate}
\item Teams alternate throwing the disc towards their opponent's bottle.
In a \emph{doubles} match, teams and teammates must alternate throws.
\item The disc is \emph{live} at the point of release and \emph{dead} upon
contacting any object afterwards.
\item Catching, blocking, or deflecting a live disc before it breaks the
$bp$ is \emph{goaltending}.
\item A dead disc may be caught, blocked, or deflected before it crosses
the $bp$.
\item The disc is \emph{catchable} if it crosses the $bp$ (i) between the
defender's knees and hairline, and (ii) within $1\mbox{ m}$ of the
bottle.

\begin{enumerate}
\item The disc is not catchable if it is within one diameter of an obstacle
when it crosses the bottle plane (safety exception).
\item The disc is not catchable if it lands within $1\mbox{ m}$ of the
$bp$ (steep toss exception).
\item If catchability is contested, the benefit of doubt usually goes to
the defender or independent third party.
\end{enumerate}
\item The disc is \emph{short} if it does not cross the $bp$.

\begin{enumerate}
\item If the disc is short, the next thrower may choose to throw from the
position where the disc comes to rest. Note that getting hit by the
disc while returning from a short position may qualify as goaltending.
\end{enumerate}
\item The disc \emph{hits} if it makes contact with the opponent's pole
or bottle, otherwise it \emph{misses}.

\begin{enumerate}
\item A hit is \emph{direct} if the disc is not dead before contacting the
pole or bottle.
\item A hit is \emph{grazing} if it does not cause the bottle to fall off
of the pole.
\end{enumerate}
\end{enumerate}

\section{Scoring}
\begin{enumerate}
\item \label{enu:dropped}Offense is awarded one point for a live catchable
disc that misses but is uncaught (\emph{dropped}). In \emph{doubles}
play the point is also awarded for live hits if the disc is uncaught.
\item No points are awarded for a grazing hit.
\item Offense is awarded two points for a pole or bottle hit, but these
are negated if Defense prevents the bottle from touching the ground.
\item \label{enu:bottle-hit}Offense is awarded an additional point for
a bottle hit, even if the bottle is caught.
\item \label{enu:stalwart}Defense gets one point if the disc and bottle
are caught by the same player after a hit, with neither touching the
ground (\emph{stalwart defense}).
\item If a throw is goaltended, the offense is awarded points corresponding
to the most likely hit.

\begin{enumerate}
\item If the goaltender does not have both feet behind the $bp$, the offense
is awarded a minimum of two points.
\end{enumerate}
\item All rules and penalties stack.
\end{enumerate}

\section{Winning the game}
\begin{enumerate}
\item \label{enu:win-conditions}Play goes to eleven points, win by two
for a \emph{singles} game and twenty-one points, win by two for a
\emph{doubles} game. 
\item A team requiring only one point to win is at \emph{game }point. If
play extends beyond eleven points teams are tied (\emph{deuce}) or
one leads by a single point (\emph{advantage}).
\item A team at game point is not awarded points for dropped discs, hitting
the bottle, or performing a stalwart defense, i.e. rules \ref{enu:dropped},
\ref{enu:bottle-hit}, and \ref{enu:stalwart} are suspended for that
team.

\begin{enumerate}
\item Points which would have been awarded instead deduct from the opponent's
HP.
\item HP do not regenerate.
\item HP are not deducted from a team unless claimed by the opposing team.
\item The game ends if either team reaches 0 HP (\emph{get it over with
already} exception).
\end{enumerate}
\item Play does not end upon loss of a point (see sections \ref{subsec:clumsiness-and-carelessness}
and \ref{subsec:traps} for scenarios), even if the conditions in
rule \ref{enu:win-conditions} are met. The penalized team may become
profoundly disadvantaged or worse under these circumstances.
\item If a bottle cracks or breaks, the owning team loses immediately.

\begin{enumerate}
\item If a bottle has only one crack which does not extend below the shoulder,
it may be replaced but not defended for the remainder of the game
(\emph{ghost bottle} exception).
\item A cracked ghost bottle cannot be replaced (\emph{no more continues}
exception).
\end{enumerate}
\end{enumerate}

\section{\label{sec:special-scenarios}Special scenarios}

\subsection{\label{subsec:clumsiness-and-carelessness}Clumsiness and carelessness}
\begin{enumerate}[leftmargin=2.8em, label=\thesubsection.\arabic*]
\item If a player steps over the $bp$ while throwing, no points are awarded
(\emph{line fault}). If this throw results in a broken bottle, the
bottle is replaced.
\item A player who's tasty beverage is hit by a live disc loses one point.
\item Dropping one's tasty beverage results in the loss of a point.
\item Grabbing an undisturbed bottle results in the loss of a point.
\item Bumping one's own pole or bottle awards the opposing team two points
for a pole bump or three points for a bottle bump, unless the bottle
is caught. This does not include bottles dropped when resetting.
\item If a team throws with an unset bottle, they must reset the bottle
before the next throw or else are penalized one point.
\item No points or penalties are given for acts of god, such as the wind
blowing the bottle off of the pole.
\end{enumerate}

\subsection{\label{subsec:traps}Traps and redemption shots}
\begin{enumerate}[leftmargin=2.8em, label=\thesubsection.\arabic*]
\item Throwing the disc in a trap results in the loss of a point.
\item The thrower is responsible for retrieving the disc from any traps
where retrieval is difficult.
\item If the trap does not have a direct line to the opponent's pole, the
thrower may opt to throw the disc back into the court from the trap
(\emph{redemption shot}).

\begin{enumerate}
\item If the redemption shot knocks the opponent's bottle to the ground,
the lost point is regained.
\item Redemption shots cannot be defended.
\item If the disc lands in any trap, the thrower loses another point.
\item Only one redemption shot is allowed per trapped throw.
\end{enumerate}
\end{enumerate}

\subsection{On fire}
\begin{enumerate}[leftmargin=2.8em, label=\thesubsection.\arabic*]
\item A player who executes three consecutive direct hits is \emph{on fire}.
The player is \emph{heating up} after the second hit.
\item When a player is on fire normal play stops. The fire player throws
until failing to make a direct hit, at which point fire is lost and
the player's fire count returns to zero.
\item Fire throws cannot be defended, the bottle may not be caught, and
points are awarded as if in normal play.
\item Consecutive hit count for a player is reset to zero if the opposing
team scores bottle points or makes a stalwart defense.
\item Consecutive hit count for a player is reset to zero if the player's
tasty beverage suffers a direct hit.
\item Goaltending does not affect any player's count.
\item Ambiguities in \emph{fire} rules may be settled by an appeal to NBA
Jam.
\end{enumerate}

\subsection{Extreme scenarios and beverage debts}
\begin{enumerate}[leftmargin=2.8em, label=\thesubsection.\arabic*]
\item A \emph{shutout} occurs when the losing team does not have a positive
score at the end of a game.

\begin{enumerate}
\item All losers of a shutout must drink an agreed upon \emph{nasty beverage}.
\item An additional drink is owed for each point the \emph{losing team}'s
score is below zero.
\end{enumerate}
\item \emph{Shooting the moon} occurs when the winning team does not have
a positive score at the end of a game (by breaking the opponents bottle).

\begin{enumerate}
\item All losers of a moon shot must drink an agreed upon \emph{nasty beverage}.
\item The loser owes a nasty beverage for each point the \emph{winner's}
score is below zero.
\end{enumerate}
\item \emph{Mutual shame} occurs if neither team has a positive score at
the end of a game.

\begin{enumerate}
\item All victims of mutual shame must drink a nasty beverage.
\item The nasty beverage must be chosen from the default list.
\item Beverage debts from mutual shame may not be wagered, forgiven, canceled,
or transferred.
\end{enumerate}
\item A \emph{photo-finish} occurs when both teams are tied at the end of
a game.
\item The nasty beverage should be chosen before the game begins.

\begin{enumerate}
\item The choice of nasty beverage may be agreed upon or changed at any
time if the players involved agree on the new choice.
\item If players cannot come to an agreement, the default beverage choices
are Bud Light with Lime, Smirnoff Margarita, or PBR with a fresh squeezed
lime.
\item In the case of a player shooting the moon, moon shots may be taken
to pay beverage debts.
\end{enumerate}
\item Beverage debts do not have to be repaid immediately.
\item Beverage debts may be wagered, forgiven, canceled, or transferred
(by trickery if necessary). Debts from mutual shame are an exception.
\item Beverage debts may be reduced if paid in bulk, at the discretion of
the parties owed.
\end{enumerate}

\section{Series and tournament play}
\begin{enumerate}[label=\thesection.\arabic*]
\item In a series, team sides and first throw alternate after each game.
\item A set is composed of an odd number of games, typically three.
\item A match is composed of an odd number of sets, typically one (equivalent
in this case).
\item Each team must use the same bottle for the duration of a match.

\begin{enumerate}
\item A broken bottle is replaced for the next game. Replacements must be
used for the rest of the match or until they break.
\end{enumerate}
\end{enumerate}
\newpage{}

\part*{Back Matter}

\section*{Credits}

Many thanks to everybody who helped with game development and codifying
the rules (in alphabetical order):\medskip{}

\begin{tabular}{>{\raggedright}p{3.5cm}>{\raggedright}p{2.5cm}}
Phillip Anderson & Julian S.\tabularnewline
Mike C. & Jon S.\tabularnewline
Will E. & Konrad S.\tabularnewline
Mike F. & Matt T.\tabularnewline
Jon K. & Kristine W.\tabularnewline
Andrew O. & Dustin W.\tabularnewline
\end{tabular}

\section*{Disclaimer}

Nobody involved with writing or distributing this document is responsible
for injuries that may be caused by playing this game.

\section*{License}

\copyright\ 2012-\currentyear\ by Phillip Anderson. 

\noindent This work is licensed under the Creative Commons Attribution
- NonCommercial - NoDerivatives 4.0 International License. To view
a copy of this license, visit \href{http://creativecommons.org/licenses/by-nc-nd/4.0/}{creativecommons.org}.
\begin{figure}[!h]
\centering{}\includegraphics[width=0.8\columnwidth]{images/by-nc-nd}
\end{figure}

\newpage{}

\ \vfill{}

\noindent For questions, comments, or other communications, please
contact or follow:

\begin{doublespace}
\noindent \faGlobe\ \href{http://andersonpa.info/polishhorseshoes}{http://andersonpa.info/polishhorseshoes}

\noindent \faEnvelopeO\  \href{mailto:brpolish@gmail.com}{brpolish@gmail.com}

\noindent \faTwitter\  \href{http://www.twitter.com/br_polish}{@br\_{}polish}

\noindent \faTwitch\  \href{http://www.twitch.tv/br_polish}{br\_{}polish}
\end{doublespace}

\end{document}
